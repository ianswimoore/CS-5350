\section{CS 6350 only: Decision Trees with Attribute Costs}
\label{q4}

\noindent[10 points] Sometimes, we may encounter situations where
the features in our learning problem are associated with costs. For
example, if we are building a classifier in a medical scenario,
features may correspond to the results of different tests that are
performed on a patient. Some tests may be inexpensive (or inflict no
harm), such as measuring the patient's body temperature or
weight. Some other tests may be expensive (or may cause discomfort
to the patient), such as blood tests or radiographs.

In this question, we will explore the problem of learning decision
trees in such a scenario. We prefer decision trees that use features
associated with low costs at the top of the tree and only use higher
cost features if needed at the bottom of the trees.  In order to
impose this preference, we can modify the information gain heuristic
that selects attributes at the root of a tree to penalize costly
attributes.

In this question, we will explore different such variations. Suppose
we denote $Gain(S, A)$ as the information gain of an attribute $A$
for a dataset $S$ (using the original version of information from
ID3). Let $Cost(A)$ denote the cost of the attribute $A$. We can
define two cost-sensitive information gain criteria for attributes
as:

\begin{enumerate}
\item $Gain_T(S, A) = \frac{Gain(S, A)^2}{Cost(A)}$

\item  $Gain_N(S, A) = \frac{2^{Gain(S, A)} - 1}{\sqrt{Cost(A) + 1}}$
\end{enumerate}

In both cases, note that attributes with higher costs are penalized
and so will get chosen only if the information gain is really high.


To evaluate these two methods for root selection, we will use the
following training set:
\begin{center}
  \begin{tabular}{llll|r}
    \hline
    Shape    & Color  & Size   & Material & Label \\ \hline
    square   & red    & big    & metal    & +     \\
    square   & blue   & small  & plastic  & +     \\
    triangle & yellow & medium & metal    & +     \\
    triangle & pink   & big    & leather  & -     \\
    square   & pink   & medium & leather  & -     \\
    circle   & red    & small  & plastic  & -     \\
    circle   & blue   & small  & metal    & -     \\
    ellipse  & yellow & small  & plastic  & -     \\
    ellipse  & blue   & big    & leather  & +     \\
    ellipse  & pink   & medium & wood     & +     \\
    circle   & blue   & big    & wood     & +     \\
    triangle & blue   & medium & plastic  & +     \\  \hline
  \end{tabular}
\end{center}

Suppose we know the following costs of the attributes:
\begin{center}
  \begin{tabular}{lr}
    Attribute & Cost \\\hline
    Shape     & 10   \\
    Color     & 30   \\
    Size      & 50   \\
    Material  & 100  \\\hline
  \end{tabular}
\end{center}

\begin{enumerate}
\item \relax[8 points] Compute the modified gains $Gain_T$ and $Gain_S$ for each
  attribute using these costs. Fill in your results in the table below. (upto 3
  decimal places)


  \begin{center}
    \begin{tabular}{l|rr}
      Attribute & $Gain_T$ & $Gain_S$ \\\hline
      Shape     &          &          \\
      Color     &          &          \\
      Size      &          &          \\
      Material  &          &          \\\hline
    \end{tabular}
  \end{center}

\item \relax[2 points] For each variant of gain, which feature would you choose
  as the root?
\end{enumerate}

\noindent (You may modify your code from the experiments to calculate these
values instead of doing so by hand.)


%%% Local Variables:
%%% mode: latex
%%% TeX-master: "main"
%%% End:
