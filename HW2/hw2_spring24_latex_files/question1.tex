\section{Warmup: Boolean Functions}
\label{sec:boolean-functions}

\begin{enumerate}
\item ~[3 points] Table \ref{tab:boolean-function-data-1} shows
  several data points (the $x$'s) along with corresponding labels
  ($y$). (That is, each row is an example with a label.) Write down
  three different Boolean functions, all of which can produce the
  label $y$ when given the inputs $x$.

  {\color{red}
    \begin{enumerate}
      \item $f_1(x): $ $y = x4$
      \item $f_2(x): $ $y = x2 \wedge x3 \wedge x4$
      \item $f_3(x): $ $y = x1' \wedge x4$
    \end{enumerate}
  }

  \begin{table}[h]
    \centering
    \begin{tabular}{ccccc}
      \toprule
      y & $x1$ & $x2$ & $x3$ & $x4$ \\
      \midrule
      0 & 0    & 1    & 1    & 0    \\
      0 & 1    & 1    & 1    & 0    \\
      1 & 0    & 1    & 1    & 1    \\
      \bottomrule
    \end{tabular}
    \caption{Initial data set}
    \label{tab:boolean-function-data-1}
  \end{table}

\item ~[5 points] Now the Table \ref{tab:boolean-function-data-1} is
  expanded to Table \ref{tab:boolean-function-data-2} by adding more
  data points. How many errors will each of your functions from the
  previous questions make on the expanded data set.

  {\color{red}
    \begin{enumerate}
      \item $f_1(x)$ has 0 errors
      \item $f_2(x)$ has 2 errors (rows 6 and 7 overall (rows 3 and 4 in table 2))
      \item $f_3(x)$ has 2 errors (rows 3 and 6 overall (row 3 in table 1, row 3 in table 2))
    \end{enumerate}
  }

  \begin{table}[h]
    \centering
    \begin{tabular}{ccccc}
      \toprule
      $y$ & $x1$ & $x2$ & $x3$ & $x4$ \\
      \midrule
      0 & 0    & 1    & 1    & 0    \\
      0 & 1    & 1    & 1    & 0    \\
      1 & 0    & 1    & 1    & 1    \\
      1 & 1    & 0    & 1    & 1    \\
      0 & 0    & 1    & 1    & 0    \\
      1 & 1    & 1    & 0    & 1    \\
      \bottomrule
    \end{tabular}
    \caption{Expanded data set}
    \label{tab:boolean-function-data-2}
  \end{table}

\item ~[5 points] Is the function in Table
  \ref{tab:boolean-function-data-2} linearly separable? If so, write
  down a linear threshold function that classifies the data. If not,
  prove that there is no linear threshold function that can classify
  the data.

  {\color{red}
    The function in Table 2 is linearly separable. The linear threshold function that classifies the data is:
    $$
    x4 = 1
    $$
    Since $x4$ alone can classify the data, it is linearly separable with a single feature. This also applies to the expanded data set with Table 1 and Table 2.
  }

\end{enumerate}

%%% Local Variables:
%%% mode: latex
%%% TeX-master: "hw2"
%%% End:
